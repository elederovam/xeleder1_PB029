\documentclass[letterpaper,12pt,oneside]{article}
\usepackage[utf8]{inputenc}
\usepackage{setspace}
\usepackage{hyperref}
\hypersetup{
    colorlinks,%
    citecolor=black,%
    filecolor=black,%
    linkcolor=black,%
    urlcolor=black
    }
\usepackage{a4wide}
\usepackage[greek, czech]{babel}

\usepackage[
	backend=biber,
           style=alphabetic,
           maxnames=5,
           alldates=iso8601]{biblatex}
\addbibresource{databaze-literatury.bib}

%\usepackage[
%   backend=biber        % if we want unicode
%  ,style=iso-authoryear % or iso-numeric for numeric citation method
%  ,autolang=other       % to support multiple languages in bibliography
%  ,sortlocale=cs_CZ     % locale of main language, for sorting
%  ,bibencoding=UTF8     % this is necessary only if bibliography file is in different encoding than main document
%]{biblatex}

%Changes the page numbers - {arabic}=arabic numerals, {gobble}=no page numbers, {roman}=Roman numerals
\pagenumbering{gobble}

%%%%%%%%%%%%%%%%% END OF PREAMBLE %%%%%%%%%%%%%%%%%%%%%

\begin{document}

\noindent  \LARGE{\textbf{Markéta Elederová}}  \\
\vspace{-2ex}



\normalsize 


\vspace{1em}


\noindent \begin{tabular}{@{} l l}
 \Large{Kontakt}    & Adresa: Chrpová 503/9, 64100, Brno  \\
     & Email: \href{mailto:elederovam@seznam.cz}{elederovam@seznam.cz} \\
     & Mobil: +420 603 397 121 \\
     & \\

 \Large{Vzdělání}    & \textbf{Masarykova univerzita} \\
     & 2015 – nyní \\
     & Fakulta informatiky \\
     & Obor: Aplikovaná informatika \\
     & \\
     & \textbf{Klasické a španělské gymnázium Brno-Bystrc} \\
     & 2007 – 2015 (maturita) \\
     & Osmileté studium (zaměření na klasické jazyky) \\
     & \\

 \Large{Pracovní}    & Brigáda: katalogizace knih dle standardu (AACR2R, MARC21) \\
  \Large{zkušenosti}   & \\
    & \\
    
 \Large{Znalosti a }   &Informatika – 3. ročník fakulty informatiky: \\
  \Large{dovednosti}   & \qquad programování: Java, základní znalost PHP, Python, C++, C \\
     &                   \qquad SQL \\
     &                   \qquad git \\
     & Cizí jazyky: \\
     & \qquad angličtina – středně pokročilý \\
     & \qquad němčina – mírně pokročilý \\
     & IQ: 160 (oficiální IQ test společnosti Mensa) \\
     & Řidičský průkaz skupiny B \\
     & \\

  \Large{Zájmy}   & Vedoucí v křesťanské organizaci pro děti a mládež Royal Rangers \\
  & Deskové hry, turistika
\end{tabular}

\newpage

\clearpage

\begin{itemize}
\item{Bible}

Bible, nebo také Písmo svaté, Boží slovo nebo Kniha knih, je základní knihou pro křesťany. Název „bible“ pochází z řeckého slova „\textgreek{βἰβλος}, což znamená „kniha“. Ve skutečnosti se však jedná spíše o celou knihovnu historických spisů, přísloví a mouder, zákonů, prorockých knih, příběhů, dopisů a mnoho dalšího od různých (často nám neznámých) autorů z různých prostředí a dob mezi více než jedním tisíciletím př. n. l. a prvním až druhým stoletím n. l.
1.1	Starý zákon
Starší část bible se nazývá Starý zákon nebo Stará smlouva, podle smlouvy Boha se svým lidem naplněné v Ježíši Kristu, který s nimi pak uzavřel novou smlouvu, z toho je zase název pro Nový zákon.
Starému zákonu se říká také „hebrejská bible“ , je to kniha společná pro křesťany i židy, v židovství se nazývá Tenak (nebo Tenach), podle počátečních písmen jejích tří částí: Tórá (Zákon), Nebím (Proroci) a Ketubím (Spisy),  které tvoří dohromady židovský, nebo tzv. palestinský kánon. V římskokatolické církvi mají k těmto knihám přidané navíc ještě další, tzv. deuterokanonické (nebo apokryfní), ale jinak Starý zákon běžně tvoří třicet devět knih, stejných jako mají židé, akorát jsou v jiném pořadí.
Starý zákon je psán hebrejsky, jen malé části (v knihách Ezdráš a Daniel) jsou aramejské.
1.2	Nový zákon
Novější, a taky o dost kratší částí bible je Nový zákon. Obsahuje dvacet sedm knih: čtyři evangelia (\textgreek{εὐαγγέλιον} = dobrá zpráva), Skutky apoštolské, dopisy (nejčastěji od apoštola Pavla různým církvím nebo svým spolupracovníkům, dále listy Janovy, Petrovy a Judův) a Zjevení Janovo, neboli Apokalypsa (\textgreek{ἀποκάλυψις} = zjevení). Původním jazykem Nového zákona je řečtina.
2	Překlady bible
Bible je jednoznačně nejpřekládanější knihou na světě. Celá bible je přeložená do více než 530 jazyků, alespoň část pak do téměř tří tisíců jazyků.  
Mezi první jazyky, do kterých byla bible přeložená, patřila řečtina a latina. Dnes už si může bibli ve svém vlastním jazyce přečíst většina lidí a na překladech do dalších jazyků se stále pracuje.
2.1	Septuaginta
2.1.1	Vznik
Podle Listu Aristeova byla Septuaginta pořízena roku 270 př. n. l., když egyptský vladař Ptolemaios II. Filadelfos pozval z Jeruzaléma do Alexandrie izraelské učence, po šesti z každého z dvanácti kmenů Izraele, takže celkem sedmdesát dva, aby přeložili Tóru z hebrejštiny do řečtiny.  Na překladu pracovali sedmdesát dva dní a výsledkem byla Septuaginta (z latiny: septuāgintā = sedmdesát), jak byla později nepřesně nazvaná podle počtu sedmdesáti dvou učenců, kteří ji překládali.
Ve skutečnosti však tento překlad vznikl pravděpodobně proto, že Izraelité v diaspoře kolem 3. století př. n. l. přestávali mluvit, a tedy i rozumět hebrejsky a potřebovali překlad do řečtiny, aby mohli rozumět svým svatým textům. Navíc na překladu nejspíš pracovali různí lidé v mnohem delším časovém období než sedmdesát dva dní.
2.1.2	Kánon
Septuaginta se jinak nazývá také „alexandrijský typ“ a byly do ní kromě kanonických knih zahrnuty i jiné spisy – knihy deuterokanonické a některé další, které uznává jen pravoslavná církev . Bylo také změněno pořadí knih, aby více chronologicky odpovídalo, a je poměrně hodně podobné pořadí, které máme v biblích dnes.
2.1.3	Další řecké překlady
Nejen odlišný kánon, ale i chyby v překladu Septuaginty vedly později k nedorozuměním mezi židy a křesťany. Ve 2. století tak vznikaly další židovské překlady do řečtiny, přesnější podle hebrejského originálu, byly to Theodotionův, Aquilův a Symmachův. V polovině 3. století vytvořil teolog Origenes Hexaplu, dílo srovnávající původní hebrejský text s různými řeckými překlady včetně Septuaginty.  
2.2	Vulgáta
2.2.1	Vznik
Nejstarším latinským překladem bible je Vetus latina, jsou to spíše neucelené překlady biblických knih z řečtiny. Časem se vyvinuly dva typy tohoto překladu, Itala a Africana. Mnich Hieronymus (Jeroným), který dostal za úkol od papeže Damasa tyto verze zrevidovat, později kolem roku 400 vytvořil nový latinský překlad přímo z hebrejštiny, Vulgátu. Jen deuterokanonické knihy Hieronymus nepřekládal tak pečlivě, některé dokonce jen převzal z Vetus latiny jak byly, nepovažoval je totiž zřejmě za tolik důležité jako ostatní.  Text Nového zákona je z větší části také převzatý.
2.2.2	Neovulgata
Neovulgata, nebo také Nova Vulgata, je revize Vulgáty pořízená římsko-katolickou církví v roce 1979. Jsou v ní spíše drobné změny a neobsahuje pseudoapokryfní knihy. Je to nyní oficiální latinský překlad římskokatolické církve. 
2.3	České překlady
2.3.1	Nejstarší české překlady
Nejstarším překladem bible na našem území byl cyrilometodějský překlad do staroslověnštiny, první překlady do češtiny vznikaly v klášterech někdy v 11. – 12. století, nejstarší rukopisná česká bible je z konce 14. století Leskovecko-Drážďanská. Překlad, který se dodnes celý zachoval, je bible Litoměřicko-Třeboňská z počátku 15. století.
Důležitým mezníkem byl vynález knihtisku, různé překlady bible patřily k prvním tištěným knihám vůbec. Mezi významné výtisky patří například Pražská bible z roku 1488, o rok mladší bible Kutnohorská, nebo bible Veleslavínská vytištěná roku 1613.  
2.3.2	Bible Kralická
Asi nejslavnějším českým překladem bible je bible Kralická, která vycházela v šesti dílech (tzv. šestidílka) v letech 1597–1594, její nejznámější vydání v jednom díle je z roku 1613. Na překladu pracovali bratři z Jednoty bratrské a bible tiskli v Kralicích, odkud je i její název.
Kralická bible je prvním překladem do češtiny z původních jazyků a nejméně po čtyři sta let byla jednoznačně nejpoužívanějším překladem u nás.
2.3.3	Moderní české překlady
Později se objevují různé další překlady bible do češtiny, nejvýznamnější je český ekumenický překlad vydaný k čtyřstému výročí bible Kralické roku 1979. S počátkem dvacátého prvního století vychází stále více nových kvalitních překladů, jako například Bible 21, Český studijní překlad, Slovo na cestu, Jeruzalémská bible a další.
\end{itemize}

\printbibliography[heading=bibintoc]

\end{document}

